\section{Auswertung}
\label{sec:Auswertung}
\subsection{Berechnung der Aktivierungsarbeit aus der Anlaufkurve}
Um die Aktivierungsarbeit der Dipole zu berechnen, werden die Messwerte für Temperatur und Strom
mit der Exponentialfunktion der Form
\begin{equation}
  y(T) = a\cdot e^{mT}
  \label{eqn:efit}
\end{equation}
gefittet.
% Mittlere heizspannung efit 3.2 0.13
\begin{figure}[H]
  \centering
  \includegraphics[width=0.8\textwidth]{plots/efit.pdf}
  \caption{Anlaufkurve des Depolarisationsstromes bei einer mittleren Heizrate von $H_1 =\SI{3.3 \pm 0.08}{\kelvin\per\minute}$.}
  \label{fig:efit1}
\end{figure}
\begin{figure}[H]
  \centering
  \includegraphics[width=0.8\textwidth]{plots/efit2.pdf}
  \caption{Anlaufkurve des Depolarisationsstromes bei einer mittleren Heizrate von $H_2 =\SI{1.26 \pm 0.07 }{\kelvin\per\minute}$}
  \label{fig:efit2}
\end{figure}
Die gefitteten Anlaufkurven sind in Abbildung \ref{fig:efit1} und \ref{fig:efit2} zu sehen.
Für den Fit der zweiten Messung wurden die in Abbildung \ref{fig:efit2} gezeigten bauen Messwerte nicht berücksichtigt.
Aus den angehängten Messwerten wurden ausserdem mittlere Heizraten für beide Messungen bestimmt. Die aus den Werten der Anlaufkurve ermittelten Heizraten lauten:
\begin{center}
  $H_1 =(3.2 \pm 0.13)\frac{\symup{K}}{\text{min}}$, $H_2 = (1.26 \pm 0.064)\frac{\symup{K}}{\text{min}}$
\end{center}
Aus der Ausgleichsrechnung ergeben sich die Fitparameter zu
\begin{center}
    $m_1 = 5680.23$, $ a_1= 24.98$, $m_2 = 5425.56$, $a_2 = 23.87$.
\end{center}
Aus dem Fitparameter $m_{(1/2)}$ werden dann nach
\begin{equation}
  W_{1/2} = m_{1/2}\cdot k_B
\end{equation}
die Austrittsarbeiten berechnet. Die berechneten Arbeiten lauten:
\begin{center}
  $W_1 = \SI{0.78e-19}{\joule}$ und $W_2 = \SI{0.75e-19}{\joule}$
\end{center}
<<<<<<< HEAD
\subsection{Berechnung der Aktivierungsenergie durch Integrieren}

Die mittleren Relaxationszeiten der Probe wurden durch \eqref{eqn:tau2} unter Zuhilfenahme riemannscher Integration berechnet und in Abb. \ref{fig:integral} dargestellt. Hierbei wurde der zweite Anstieg jeweils nicht betrachtet.

\begin{figure}
  \centering
  \includegraphics{./plots/integral.pdf}
  \caption{Mittlere Lebensdauern $\tau(T)$ der Probe in Abhängigkeit der Temperatur $T$.}
  \label{fig:integral}
\end{figure}

Eine lineare Ausgleichsrechnung von $\ln(\tau(T)\cdot H)$ augetragen gegen $1/T$ (siehe Abb. \ref{fig:log}) liefert
\begin{center}
  $m_1 = \SI{7417 \pm 426}{\kelvin}$ und $b_1 = -25.5\pm1.6$
\end{center}
für die erste Messung. $m$ bezeichnet hierbei die Steigung der Geraden, $b$ den Ordonatenabschnitt.
Es ergibt sich hierdurch
\begin{equation*}
  W_1^i = k_b m_1 = \SI{102 \pm 6}{\zepto\joule}.
\end{equation*}

\begin{figure}
  \includegraphics{./plots/log.pdf}
  \caption{}
  \label{fig:log}
\end{figure}

Für die zweite Messreihe ergibt sich entsprechend:
\begin{center}
  $m_2 =  \SI{6713 \pm 324}{\kelvin}$, $b_2=-23.7 \pm 1.3$, $W_2^i= \SI{93 \pm 4}{\zepto\joule}$
\end{center}

\subsection{Bestimmung der charakteristischen Relaxationszeit}
Als Maxima wurden gewählt:
\begin{center}
  $T_{max,1} = \SI{-8.8}{\celsius}$ und $T_{max,2} = \SI{-16.7}{\celsius}$
\end{center}
Nach Verrechnung dieser Werte, sowie $W_1$ bzw. $W_2$, ergibt sich gemäß \eqref{eqn:tau3}:
\begin{center}
  $\tau_{max,1}=  \SI{177\pm 10}{\second}$ und $\tau_{max,2}=  \SI{466\pm23}{\second}$
\end{center}

Dieser Werte lassen sich wiederrum durch \eqref{eqn:relaxation} in $\tau_0$ überführen.
Dies liefert
\begin{center}
  $\tau_{0,1}= \SI{0.12 \pm 0.19}{\nano\second}$ und $\tau_{0,2} = \SI{2.0 \pm 2.6}{\nano\second}$.
\end{center}
||||||| merged common ancestors
=======
In den Tabellen \ref{tab:daten1} und \ref{tab:daten2} sind die gemessenen Werte für Depolarisationsstrom und Temperatur zu sehen.
\begin{table}[H]
  \centering
  \caption{Aufgenommene Werte für den Depolarisationsstrom in Abhängigkeit von der Temperatur bei einer Heizrate von $H_1 =(3.2 \pm 0.13)\frac{\symup{K}}{\text{min}}$}
   \label{tab:daten1}
  \begin{tabular}{c|c|c|c}
    I in pV& T in K&I in pV& T in K\\
    \hline
    -34.4& -6.1&24.6 &-12.25\\
    -32& -4.8&26 &-12.5\\
    -28.8& -5.4&28 &-13\\
    -25.3& -12&29.8 &-14\\
    -18.3& -15.5&31.5 &-15\\
    -15& -21.5&33.3 &-16\\
    -11.8& -26.0&37& -20\\
    -8.8 &-29.25&38.8& -20\\
    -5.8 &-27&   38.8& -22\\
    -3.0 &-20.5& 40.8 &-23.5\\
    -0.1 &-13.5& 42.8 &-24.8\\
    2.4 &-9.8&   44.8 &-25.5\\
    5.0 &-8&     46 &-26.6\\
    7.5 &-8&     48.2& -25\\
    9.9 &-8.5&   50 &-24\\
    12 &-9&      51.9& -22.25\\
    14.2& -9.5&  53.3& -20.25\\
    16.2 &-10&   55.6& -18\\
    18.5 &-10.5& 57.5& -13.5\\
    20.3 &-10.8& 60.3& -11.5\\
    22.6 &-11.25&&\\

  \end{tabular}

\end{table}
\begin{table}[H]
\centering
  \caption{Aufgenommene Werte für den Depolarisationsstrom in Abhängigkeit von der Temperatur bei ei}
  \label{tab:daten2}
  \begin{tabular}{c|c|c|c}
I in pV& T in K&I in pV& T in K\\
    \hline
    -87.5& -1.2&4.1& -3.9\\
    -83.9& -1.6& 5.0& -4.1\\
    -78.1& -1.4&6.1& -4.2\\
    -73.8& -1.2&7.1& -4.3\\
    -69.6& -0.9&8.1& -4.5\\
    -64.7& -0.8&9.3& -4.6\\
    -60.5& +0.8&10.4& -4.8\\
    -56.3& +0.3&11.6& -5.1\\
    -52.4& +0.8&12.7& -5.3\\
    -50.0& -2.5&14& -5.5\\
    -47.1& -2.6&15.1& -5.7\\
    -44.6& -2.4&16.2& -5.9\\
    -42.5& -2.6&17.2& -6.0\\
    -40.8& -2.9&18.2& -6.2\\
    -39.2& -3.2&19.2& -6.4\\
    -37.9& -2.0&20.2& -6.5\\
    -36& -2.7&21.2& -6.7\\
    -34.7& -2.9&22.0& -6.9\\
    -33& -3.0&22.9& -7.1\\
    -32& -3.4&22.9& -7.1\\
    -30.8& -3.8&23.8& -7.4\\
    -29.2& -4.4&24.9& -7.7\\
    -28.1& -4.9&25.8& -8.1\\
    -27.1& -5.5&26.7& -8.5\\
    -25.8& -6.2&27.7& -8.9\\
    -24.5& -7&28.7& -9.7\\
    -23.3& -7.9&29.6& -9.9\\
    -22.2& -8.7&30.6& -10.5\\
    -21.0& -9.6&31.4& -10.8\\
    -19.9& -10.5&32.2& -11\\
    -18.8& -12.0&33.3& -11.8\\
    -17.6& -12.5&34.4& -12.0\\
    -16.7& -12.5&35.3& -12.5\\
    -15.5& -12.0&36.3& -13.25\\
    -14.6& -10.8&37.2& -14\\
    -13.4& -10&38.2 -13.8\\
    -12.6& -9&39.1& -14\\
    -11.5& -8.25&40.2& -14\\
    -10.7& -7.25&41.1& -14\\
    -9.8& -6.25&41.8& -13.8\\
    -8.9& -5.5&42.7& -13.5\\
    -8.1& -4.5&43.7& -13.5\\
    -7.1& -4.0&44.8& -13.4\\
    -6 &-4&46.1& -13.2\\
    -4.4& -3.5&47.1& -13\\
    -3.2 &-3.5&48.3& -12.5\\
    -2.1 &-3.5&49.4& -12.2\\
    -1.2 &-3.5&50.7& -11.2\\
    -0.2 &-3.5&51.8& -12\\
    0.8 &-3.5&52.8& -10.5\\
    1.8 &-3.6&53.8& -10\\
    3.0 &-3.8&53.8& -10\\
    &&54.8& -9.2\\
    &&55.8& -8.5\\
  &&  56.8& -8\\
  &&  57.7& -7.5\\
  &&  58.8& -6.7\\
  &&  59.6& -6.3\\

  \end{tabular}
\end{table}
>>>>>>> Tabellen Dipolrelaxation in Auswertung

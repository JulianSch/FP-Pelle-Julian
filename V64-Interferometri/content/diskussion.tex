\section{Diskussion}
Die Abweichung des experimentell bestimmten $n =1.814 \pm 0.244$ vom erwarteten $n_{Herst} = 1.5$ stellt mit $\Delta_{Glas} = 0,109 \pm 0,070$, bzw. einer relativen Abweichung von $\frac{\Delta{Glas}}{n_{Herst}-1} = 0,311$, einen vertretbaren, wenn auch nicht geringen Fehler dar. Ursachen für diese Abweichung können in der Bedienung der feinen Stellschraube oder trotz verwendeter Abschirmung auftretender Luftzüge im Interferometer liegen.

Die Messung des Brechungindexes von Luft ergibt mit konstant $n_{mess}=1.006298$ bei $p=\SI{995}{\milli\bar}$ solide Ergebnisse, welche allerdings über dem Theoriewert von $n_{theo}=1,000272$ \cite{Luftdruck} liegen. Vergleicht man $n_{mess}-1$ und $n_{theo}-1$, so ergibt sich eine absolute Abweichung von $\Delta_{n-1} = 0,006026$, bzw. eine relative Abweichung von $\frac{\Delta_{n-1}}{n_{theo}-1}=22,151441$.

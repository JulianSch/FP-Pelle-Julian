\section{Diskussion}
Offensichtlich sind die starken Schwankungen bei der Messung des Brechungindexes der Glasplatten zwischen $0,7590$ und $1.034$ \ref{tab:plattenindex}, sowie deren dramatische Abweichung vom Theoriewert von $1,35$. Vernachlässigt man den stark aus der Reihe fallenden Wert von $0,75$, welcher vermutlich durch Luftzüge und damit auf sehr geringem Raum schwankendem Luftdruck innerhalb des Interferometers verursacht wurde, so bleibt dennoch die Abweichung von $n_1$ vom Theoriewert in drei Größenordnungen.

Die Messung des Brechungindexes von Luft ergibt mit konstant $n_{mess}=1.006298$ bei $\SI{995}{\milli\bar}$ solide Ergebnisse, welche allerdings über dem Theoriewert von $n_{theo}=1,000272$ \cite{Luftdruck} liegen. Vergleicht man $n_{mess}-1$ und $n_{theo}-1}$, so ergibt sich ein Abweichung von $\Delta_{n-1} = 0,006026$.





	\section{Durchführung}

  \begin{figure}[H]
    \center
    \includegraphics[width=\textwidth]{./plots/Versuchsaufbau.JPG}
    \label{fig:aufbau}
    \caption{Aufbau des Sagnac-Interferometers}
  \end{figure}
  \begin{figure}[H]
    \center
    \includegraphics[width=\textwidth]{./plots/Strahlengang.JPG}
    \label{fig:bauteilpos}
    \caption{Strahlengang des Sagnac-Interferometers und Montagepunkte der Lochblenden zur Justage}
  \end{figure}
  In den Abbildungen \ref{fig:aufbau} und \ref{fig:bauteilpos} ist der generelle Aufbau des Sagnac Interferometers zu sehen. Hinter dem PBSC aus der Abbildung wird
  zur Messung der Interferenzmuster noch ein weiterer PBSC zur Teilung des Strahls auf 2 Photodioden eingebaut. Die Signale der Dioden werden auf einem Oszilloskop von einander subtrahiert und
  als Differenzsignal visualisiert.
  Zur zusätzlichen Justierung wird der HeNe-Laserstrahl vor dem Auftreffen auf den PBSC noch über 2 Spiegel geleitet. Durch den linken unteren Spiegel kann die Aufteilung des Strahls in hin- und rückläufigen
  Strahl bewirkt werden, was für die späteren Messungen von nöten ist. Zur korrekten Justierung des Strahls auf die Spiegel werden ausserdem Lochblenden verwendet.\\
  Bevor mit der Messung der Intensitätsmaxima und Minima begonnen werden kann, werden nun die die Strahlen im Interferometer korrekt auf die Photodiode
  ausgerichtet.\\
  Anschließend werden die Plättchen in das Interferometer gebracht. Nun wird der Polarisationsfilter, welcher vor Punkt 1 positioniert ist,
  in 10\circ Schritten gedreht. Aus dem auf dem Oszilloskop sichtbaren Referenzsignal lassen sich nun die maximalen und minimalen Intensitäten ablesen, welche entstehen, wenn
  man die Plättchen im Interferometer um einen Winkel im Strahl kippt. Aus den gemessenen Daten wird in der Auswertung der Kontrast errechnet.\\
  \\
  Im zweiten Teil des Versuchs soll der Brechungsindex der zuvor bereits verwendeten Plättchen ermittelt werden. Dazu werden die um 10\circ aus der Ebene senkrecht zum Strahl gekippten Plättchen in Laserstrahl des Interferometers eingeführt.
  Anschließend wird die Anzahl der Interferenzmaxima bei einer Rotation um 10 - 11 Grad senkrecht zur Strahlebene in 1 Grad Schritten gemessen.\\
  \\
  Im letzten Teil des Versuchs wird statt der Plättchen eine Gaszelle in den Laserstrahl geschoben und mithilfe einer angeschlossenen Vakuumpumpe evakuiert.
  Hier werden die Anzahl der Counts bei zurück auf Umgebungsdruck steigendem Druck in der Kammer aufgenommen.


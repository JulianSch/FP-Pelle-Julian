\section{Auswertung}
\label{sec:Auswertung}
Um die Horizontalkomponente des Erdmagnetfeldes zu bestimmen, werden die zur erzeugung der Felder gemessenen Ströme zunächst mithilfe der Formel
\begin{equation}
  B = \mu_0 I \frac{N}{l}
  \label{eqn:Spulenbfeld}
\end{equation}
in magnetische Feldstärken umgerechnet.
Die dafür benötigten Daten werden der Anleitung\cite{Anleitung} entnommen und lauten:
\begin{table}[H]
  \centering
  \caption{Herstellerangaben für Windungszahl $N$ und Radius $r$ der drei Verwendeten Spulenpaare}
  \label{tab:Spulendaten}
  \begin{tabular}{ccc}
    Spule & $N$ &$r$ in cm\\
    \hline
    Horizontalfeld& 151 & 15.79\\
    Vertikalfeld  & 20  & 11.735\\
    RF-Feld       & 11  & 16.39\\
  \end{tabular}
\end{table}
Zur Bestimmung der Horizontalkomponente werden nun die Werte für B-Feld und Frequenz aus Tabelle \ref{tab:messung} je linear gefittet.
\begin{table}
  \caption{Werte beider Messungen für horizontales Magnetfeld in Abhängigkeit von der Frequenz}
  \label{tab:messung}
  \begin{tabular}{|c|c|c|c|}
    $f_1$ in kHz & $B_{H1}$ in $\mu$T &$f_2$ in kHz & $B_{H2}$ in $\mu$T\\
    \hline
     98& & 109& \\
     194& & 210&\\
     296& & 303&\\
     393& & 400&\\
     499& & 507&\\
     608& & 613&\\
     706& & 707&\\
     802& & 805&\\
     916& & 901&\\
     1016& & 1027&\\
  \end{tabular}
\end{table}

\section{Auswertung}
\label{sec:Auswertung}
\subsection{Bestimmung der Horizontalfeldkomponente des Erdmagnetfelds}
\label{subsec:erdbfeld}
Um die Horizontalkomponente des Erdmagnetfeldes zu bestimmen, werden die zur erzeugung der Felder gemessenen Ströme zunächst mithilfe der Formel
\begin{equation}
  B = \mu_0 I \frac{N}{l}
  \label{eqn:Spulenbfeld}
\end{equation}
in magnetische Feldstärken umgerechnet.
Die dafür benötigten Daten werden der Anleitung\cite{Anleitung} entnommen und lauten:
\begin{table}[H]
  \centering
  \caption{Herstellerangaben für Windungszahl $N$ und Radius $r$ der drei Verwendeten Spulenpaare}
  \label{tab:Spulendaten}
  \begin{tabular}{c|c|c}
    Spule & $N$ &$r$ in cm\\
    \hline
    Horizontalfeld& 151 & 15.79\\
    Vertikalfeld  & 20  & 11.735\\
    RF-Feld       & 11  & 16.39\\
  \end{tabular}
\end{table}
Zur Bestimmung der Horizontalkomponente werden nun die Werte für B-Feld und Frequenz aus Tabelle \ref{tab:messung} je linear gefittet.
\begin{table}
  \centering
  \caption{Werte beider Messungen für horizontales Magnetfeld in Abhängigkeit von der Frequenz}
  \label{tab:messung}
  \begin{tabular}{|c|c|c|c|}
    $f_1$ in kHz & $B_{H1}$ in $\mu$T &$f_2$ in kHz & $B_{H2}$ in $\mu$T\\
    \hline
     98& & 109& \\
     194& & 210&\\
     296& & 303&\\
     393& & 400&\\
     499& & 507&\\
     608& & 613&\\
     706& & 707&\\
     802& & 805&\\
     916& & 901&\\
     1016& & 1027&\\
  \end{tabular}
\end{table}
\subsection{Berechnung der Landé-Faktoren und der Kernspins}
\label{subsec:lande}
Um die Landé-Faktoren auszurechnen, wird Formel \eqref{??} aus der Theorie verwendet. In diese Formel wird nun für die Energiedifferenz die Energie der Lichtquanten eingesetzt, die mithilfe der RF-Spule erzeugt werden. Mit der Energie der Photonen $U = h\cdot f$, wobei f die Frequenz und h die Planckkonstante sind, lässt sich die Formel \eqref{??} jetzt zu
\begin{equation}
  g_F = \frac{h\cdot f}{\mu_B\cdot B}
  \label{eqn:landefaktor}
\end{equation}
umstellen. In diese Gleichung wird nun Steigung $m = \frac{B}{f}$ des Fits aus \ref{subsec:erdbfeld} eingesetzt wodurch sich die Landé-Faktoren zu
\begin{center}
  $g_{F1} = $ $g_{F2} = $ $g_{F3} = $ $g_{F4} = $
\end{center}
ergeben.\\
Für die Berechnung der Kernspins der beiden Rubidium Isotope wird die Formel \eqref{???} nach $I$ umgestellt. Der Elektronenhüllenspin $J = \frac{1}{2}$ eines Alkaliatoms lässt sich aus der Summe der Spinquantenzahl $S = \frac{1}{2}$ und der Drehimpulsquantenzahl $L = 0$ berechnen, woraus $g_J = 2.0023$ folgt. Die Formel \eqref{???} verändert sich dann zu
\begin{equation}
  I = \frac{\frac{g_J}{g_F}-1}{2}
  \label{eqn:kernspin}
\end{equation}.
Die Kernspins, die sich aus den beiden Messungen ergeben lauten dann:
\begin{center}
  $I_1 =$$I_1 =$$I_1 =$$I_1 =$
\end{center}

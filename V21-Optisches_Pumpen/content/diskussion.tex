\section{Diskussion}
\label{sec:Diskussion}
Aus den linearen Regressionen in Abschnitt \ref{erdfeld}
In Tabelle \ref{tab:spin} sind die errechneten Kernspins und dazu recherchierte Literaturwerte dargestellt.\\
\begin{table}[H]
  \centering
  \caption{Vergleich von errechneten Kernspins mit Literaturwerten\cite{2}.}
  \label{tab:spin}
  \begin{tabular}{c|c|c}
    ermittelter Wert & Literaturwert & relative Abweichung\\
    \hline
    $\symup{I_{1A}} = 1.34 \pm 0.05$& 1.5& 10.6\%\\
    $\symup{I_{1B}} = 2.33 \pm 0.05$& 2.5& 6.8\%\\
    $\symup{I_{2A}} = 1.38 \pm 0.10$& 1.5& 8\%\\
    $\symup{I_{2B}} = 2.34 \pm 0.09$& 2.5& 6.4\%\\
  \end{tabular}
\end{table}
Aus den errechneten Kernspins kann man erkennen, dass es sich bei $\symup{I_{1A}}$ und $\symup{I_{2A}}$ um das Isotop $\symup{^{85}Rb}$ und bei $\symup{I_{1B}}$ und $\symup{I_{2B}}$ um das Isotop $\symup{^{87}Rb}$ handelt.
Die Abschätzungen des quadratischen Zeeman-Effekts ergeben um minimal 5 Größenordnungen kleinere Werte als der lineare Zeeman-Effekt (Tabelle \ref{tab:quad}). Daher sind durch Vernachlässigung des quadratischen Terms kaum Abweichungen zu erwarten.
Die ermittelten Kernspins weisen nur sehr kleine Abweichungen von den Literaturwerten auf.
Aus dem ermittelten Isotopenverhältnis \ref{subsec:iso} lässt sich schließen, dass das Isotopenverhältnis deutlich von dem natürlich vorkommenden abweicht.
Weitere Fehler in den Messwerten sind vor allem durch die hohe Lichtempfindlichkeit der Apparatur trotz der Abdeckung und durch unzureichende Kompensation des Erdmagnetfeldes zu erwarten.

\section{Diskussion}
Das negative Vorzeichen der Magnetfelder in \ref{subsec:erdbfeld} resultiert aus der Antiparallelität zwischen dem Horizontalfeld der Spule und der Erde.
\begin{table}[H]
  \centering
  \caption{Alle zu berechnenden Werte und ggf. recherchierte Literaturwerte.}
  \label{tab:ergebnisse}
  \begin{tabular}{c|c|c|c}
    &ermittelter Wert & Literaturwert & relative Abweichung\\
    \hline
    \hline
    Erdmagnetfeld& $B_{h12} = \SI{-21.12 \pm 9.08}{\micro\tesla}$ & $B_{\symup{Lit}}=\SI{19.34}{\micro\tesla}$ \cite{1} & 9.20\%\\
    & $B_{h22} = \SI{-24.48 \pm 9.44}{\micro\tesla}$ & $B_{\symup{Lit}}=\SI{19.34}{\micro\tesla}$ \cite{1} & 26.58\% \\
    \hline
    Landè-Faktor&$\symup{g_{F1}} = 0.6 \pm 0.07$ &-- &-- \\
    &$\symup{g_{F2}}=0.328\pm0.022$&--&--\\
    &$\symup{g_{F3}}=0.55\pm0.06$&--&--\\
    &$\symup{g_{F4}}=0.327\pm0.023$&--&--\\
    \hline
    Kernspins&$\symup{I_1}=1.18\pm0.20$&1.5\cite{2}&21.33\%\\
    &$\symup{I_2}=2.56\pm0.20$&2.5\cite{2}&2.4\%\\
    &$\symup{I_3}=1.32\pm0.20$&1.5\cite{2}&13.64\%\\
    &$\symup{I_4}=2.56\pm0.21$&2.5\cite{2}&2.4\%\\
  \end{tabular}
\end{table}
 In Tabelle \ref{tab:ergebnisse} sind alle zu berechnenden Größen mit ihren relativen Abweichungen zu sehen.
Fehler in den Messwerten sind vor allem durch die hohe Lichtempfindlichkeit der Apparatur trotz der Abdeckung zu erwarten. Weitere mögliche Quellen sind zum einen das genaue Einstellen des Frequenzgenerators, sowie genaues Ablesen der Minima auf dem Oszilloskop.
\label{sec:Diskussion}
\subsection{Relativer Fehler}
Alle relativen Fehler wurden nach der Formel
\begin{equation*}
  \tilde{x} = \frac{ \lvert x_{lit} - x_{mess} \rvert}{\lvert x_{lit} \rvert}
  \cdot 100 \%
\end{equation*}
berechnet, dabei bezeichnet $x_{lit}$ den Literaturwert der Messgröße $x_{mess}$.

\section{Diskussion}
\label{sec:Diskussion}
\subsection{Relativer Fehler}
Alle relativen Fehler wurden nach der Formel
\begin{equation*}
  \tilde{x} = \frac{ \lvert x_{lit} - x_{mess} \rvert}{\lvert x_{lit} \rvert}
  \cdot 100 \%
\end{equation*}
berechnet, dabei bezeichnet $x_{lit}$ den Literaturwert der Messgröße $x_{mess}$.

\subsection{Mittlere Lebensdauer der Myonen}
Beim Vergleich der bestimmten mittleren Lebensdauer von $\tau = \SI{2.03 \pm 0.09}{\micro\second}$ mit dem Theoriewert von $\SI{2.1969811 \pm 0.0000022}{\micro \second}$ \cite{PDG} ergibt sich ein relativer Fehler von $\delta_{\tau} = (8 \pm 4) \, \%$.  Zu dieser Berechnung wurde das Python-Paket $Uncetainties$ \cite{uncertainties} verwendet. Diese Abweichung stellt zwar ein Ergebnis dar, welches den Einsatz der verwendeten Aparatur legitimiert, allerdings dürfte sich die Unsicherheit durch eine längere Messzeit weiter verringern lassen. Zu beachten ist hierbei, dass der Theoriewert freie Myonen beschreibt, allerdings werden in diesem Versuch Myonen im Szintillatormaterial gemessen. Hierbei kann es passieren, dass ein Myonen von einem Atom eingefangen wird, d.h. es bildet sich ein sogenanntes $\mu$-Atom. Da die Lebensdauer gebundener Myonen höher ist als die freier, bietet dieser Effekt eine Erklärung der gemessenen längeren Lebensdauer.
 Die reale Messzeit betrug $\SI{96454}{\second}$, was etwa $\SI{27}{\hour}$ entspricht.
Zudem könnte eine Optimierung der Aparatur das Verhältnis aus Start- und Stopsignalen verbessern. So könnte z.B. die Wahl eines anderen Szintillatormaterials zu einer höheren zeitlichen Auflösung führen. Die zwar geringe Totzeit von $t_{tot} = \SI{0.09}{\micro \second}$ sorgt, da sie im Bereich der höchsten Ereignisrate liegt (vgl. Abb. \ref{fig:tau}), zu einem hohen vermuteten Ereignisverlust.

Die Totzeit könnte auch zum Stopsignalverlust beitragen, da der Impulszähler zum Zeit-Amplituden-Converter (ZAC) parallel geschaltet ist. Wenn der ZAC über eine eigene Totzeit verfügt, kann er zu einem Ereignisverlust führen. Da der Impulszähler nicht Start- und Stopsignale verarbeitet, sondern nur die weiter auseinander liegenden Stopsignale, würde hier eine Totzeit weniger ins Gewicht fallen.

\subsection{Untergrund}

Der durch Ausgleichsrechnung berechnete Untergrund beträgt $U_{mess} = 1.9 \pm 1.2$. Dies entspricht zwar dem Erwartungswert $U_{Erwartung} \approx 1.78$. Allerdings ist durch die hohe Unsicherheit von $U_{mess}$ von ca. $63\,\%$ keine weiter belastbare Untersuchung möglich.

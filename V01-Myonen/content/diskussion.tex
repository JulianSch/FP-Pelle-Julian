\section{Diskussion}
\label{sec:Diskussion}
\subsection{Relativer Fehler}
Alle relativen Fehler wurden nach der Formel
\begin{equation*}
  \tilde{x} = \frac{ \lvert x_{lit} - x_{mess} \rvert}{\lvert x_{lit} \rvert}
  \cdot 100 \%
\end{equation*}
berechnet, dabei bezeichnet $x_{lit}$ den Literaturwert der Messgröße $x_{mess}$.

\subsection{Mittlere Lebensdauer der Myonen}
Beim Vergleich der bestimmten mittleren Lebensdauer von $\tau &= \SI{2.03 \pm 0.09}{\micro\second}$ mit dem Theoriewert von $\SI{2.1969811 \pm 0.0000022}{\micro \seconds}$ \cite{PDG} ergibt sich ein relativer Fehler von $\delta_{\tau} = 8 \pm 4 \, \%$.  Zu dieser Berechnung wurde das Python-Paket $Uncetainties$ \cite{uncertainties} verwendet. Diese Abweichung stellt zwar ein Ergebnis dar, welches den Einsatz der verwendeten Aparatur legitimiert, allerdings dürfte sich der Fehler durch eine längere Messzeit weiter verringern lassen. Die reale Messzeit betrug $\SI{96454}{\second}$, was etwa $\SI{27}{\hour}$ entspricht.
Zudem könnte eine Optimierung der Aparatur das Verhältnis aus Start- und Stopsignalen verbessern. Die zwar geringe Totzeit von $t_{tot} = \SI{0.09}{\micro \seconds}$ sorgt, da sie im Bereich der höchsten Ereignisrate liegt (vgl. Abb. \ref{fig:tau}).

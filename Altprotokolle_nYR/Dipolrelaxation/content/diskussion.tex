\section{Diskussion}
\label{sec:Diskussion}
Werden die auf die beiden verschiedenen Methoden bestimmten
Aktivierungsenergien wie in Tabelle \ref{tab:verg} miteinander verglichen,
lässt sich die Erwartung, dass Methode 2 durch den Einbezug der gesamten
Messkurve bessere Ergebnisse liefert, verifizieren. In der Tabelle steht $W_{a}$ für
die mit Methode 1 bestimmte Schwellenenergie und $W_{b}$ für selbige mit Methode 2 bestimmt. Die prozentuale Abweichung
bezieht sich auf den Literaturwert.
\begin{table}[H]
  {\tiny
  \caption{Vergleich von Literatur \cite{V48a} und Messwerten}
  \label{tab:verg}
    \begin{tabular}{S S| S S| S S}
    \toprule
  $ \text{Heizrate}  \: / \: \si{\kelvin\per\minute}$
  & $ \text{Literaturwert}  \: / \: \si{\electronvolt}$
  & $W_{b} \: / \: \si{\electronvolt}$
  & $\text{Abweichung} \: / \: \% $
  & $W_{a} \: / \: \si{\electronvolt}$
  & $\text{Abweichung} \: / \% $ \\
\midrule
1.5 & 0.66 & 0.75 $\pm$ 0.02 & 14 & 0.53 $\pm$ 0.01 & 20 \\
3.0 & 0.66 & 0.83 $\pm$ 0.02 & 26 & 0.53 $\pm$ 0.02 & 20 \\
\bottomrule
\end{tabular}
}
\end{table}
Darüberhinaus fällt auf, dass die Werte der für die beiden verschiedenen Heizraten sich bei der zweiten Methode nur in ihrem
Fehler unterscheiden, was von einer gelungenen Wahl der Heizspannung zeugt. Auch die Abweichungen von
lediglich maximal $26\%$ spricht für einer gelungenen Versuchsdurchführung ohne große Fehlerquellen.
Die teilweise höheren Abweichungen bei Methode 1 lassen sich zum einen durch die schon beschriebenen Fakt erklären,
dass es sich bei dieser Methode lediglich um eine Näherungsmethode handelt und darüberhinaus der Bereich der Näherungsmethode subjektiv gewählt wurde.
Da analoge Messinstrumente verwendet wurden kann es zu Ablesungsfehlern gekommen sein, die ob der guten Ergebnisse
aber keine große Rolle gespielt haben.
Zu Beginn der Messungen kam es zu großen Varianzen des Depolarisationsstroms, weshalb diese Werte in der Auswertung nicht berücksicchtigt wurden.
Im Gegensatz zu den gemessenen Ativierungsenergien weisen die berechneten Relaxationszeiten
\begin{align*}
  \tau_{0_1} &= \SI{3(2)e-10}{\second} \text{ und}\\
  \tau_{0_2} &= \SI{42(1)e-11}{\second}
\end{align*}
mit einer Differenz von 3 bzw. 4 Größenordnungen einen deutliche Abweichungen vom Literaturwert \cite{V48a}
\begin{equation*}
  \tau_0 = \SI{4e-14}{\second}
\end{equation*}
auf.
Auf eine Bestimmung der relativen Abweichung wird auf Grund dieser großen Abweichung verzichtet.
Gründe für diese Differenz zwischen Literatur- und Messwerten könnte das exponentielle Verhältnis zwischen
der Aktivierungsenergie $W$ und der Relaxationszeit $\tau_0$ sein (siehe Formel \ref{eqn:tau}).
So wirkt sich eine kleine Schwankung in der Aktivierungsenergie deutlich in der Berechnung der Relaxationszeit
aus.
Jedoch kann dies alleine nicht die Abweichung um minestens 3 Größenordnungen erklären, sondern spricht dafür, dass
im Experiment und / oder in der Auswertung Vereinfachungen angenommen wurden oder Fehler aufgetreten sind.
